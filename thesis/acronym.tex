\iftoggle{german}{
\def\AbbrvName{Abkürzungsverzeichnis}
}{
\def\AbbrvName{List of Abbreviations}
}
\section*{\AbbrvName}
\addcontentsline{toc}{section}{\AbbrvName}
\sectionmark{\AbbrvName}

\def\LongestAcro{wmSDN}
\begin{acronym}[\LongestAcro] % Pass the longest acronym here to align them correctly
\setlength{\itemsep}{-\parsep} % reduce the space between acronyms
 \acro{TA}{Test Automation}
 \acro{OS}{Operating System - Betriebssystem}
 \acro{BPMN}{Business Process Model and Notation - Eine grafische Spezifikationssprache in der Wirtschaftsinformatik und im Prozessmanagement}
 \acro{ISTQB}{International Software Testing Qualifications Board (http://www.istqb.org/)}
 \acro{XML}{„Extensible Markup Language“ - erweiterbare Auszeichnungssprache zur Darstellung hierarchisch strukturierter Daten in Form von Text}
 \acro{AE}{Automation Engine - Bestandteil der CA Automic Workload Automation }
 \acro{AWA}{CA Automic Workload Automation}
 \acro{VARA}{Variablen-Objekt der AWA, welches einen Schlüsselwert mit fünf dazugehörigen Werten speichert}
  \acro{b4A}{Kurzform für die Software b4Automic Solution}
 \acro{JOB}{Ausführbares-Objekt der AWA, was benutzt wird um z.B. OS Operationen auszuführen}
 \acro{USER}{Benutzer-Objekt der AWA}
\end{acronym}

